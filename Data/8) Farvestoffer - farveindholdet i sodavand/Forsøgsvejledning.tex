\documentclass[10pt,a4paper]{article}
\usepackage[utf8]{inputenc}
\usepackage[danish]{babel}
\usepackage{amsmath}
\usepackage{amsfonts}
\usepackage{amssymb}
\usepackage{graphicx}
\usepackage[left=2cm,right=2cm,top=2cm,bottom=2cm]{geometry}
\author{Sebastian Hansen og Mikkel B. Goldschmidt}
\title{Bestemmelse af farvestoffers koncentration i sodavand}
\begin{document}
\maketitle

\section{Formål}
At bestemme indholdet af de to forskellige farvestoffer der er indeholdt i en discountsodavand.

\section{Teori}
Lambert Beers lov. 

\section{Materialer}
\begin{itemize}
	\item Spektrofotometer
	\item Kuvetter
	\item Demineralisreret vand
	\item Farvestof E160a
	\item Farvestof E141
	\item 2 Bægerglas
	\item Grøn sodavand
\end{itemize}


\section{Fremgangsmåde}

\begin{enumerate}
	\item Fremstil standardkurver for hvert af de to farvestoffer.
		\begin{enumerate}
			\item Fremstil først en opløsning med farvestoffet med $1\frac{mg}{L}$.
			\item Bestem da hvilken bølgelængde der absorberes mest ved og bestem absorbansen ved denne bølgelængde.\label{item:bestembolge}
			\item Fortynd opløsningen faktor 5, og gentag sidste del af punkt \ref{item:bestembolge}. Gentag dette minimum 10 gange.
		\end{enumerate}
	\item Bestem absorbans ved hver bølgelængder fra punkt \ref{item:bestembolge}.
\end{enumerate}

\section{Databehandling}
Først fremstiller vi standardopløsninger af de to stoffer $\beta$-caroten (E160a) og clorophyl-kobber-koplex (E141). Vi fremstiller opløsninger med koncentrationer på henholdsvis
\pagebreak



\end{document}
