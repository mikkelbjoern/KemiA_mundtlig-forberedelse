\textbf{Spørgsmål}\\
Med udgangspunkt i eksperiment ”Farvestofindholdet i sodavand” skal du redegøre for, hvorledes man i praksis kan bestemme farvestofindholdet i en sodavand. 

I din gennemgang kan du gøre brug af nedenstående stikord. 

I din besvarelse skal bilagene inddrages.


\vspace{0.5 cm}
\textbf{Stikord}\\
Lambert-Beers lov, farvedannelse i organiske molekyler, orbitaler og hybridisering.

\section*{Synopsis}

\subsection{Farvedannelse i organiske molekyler}
Organiske molekyler tager farve af flere forskellige ting.
Primært tager de farve når der dannes en delokaliseret elektronsky rundt om molekylet. 

\paragraph{Kojugerede dobbeltbindinger} er en af de ting der kan danne en delokaliseret elektronsky. 
Konjugerede dobbeltbindinger er skiftende dobbelt- og enkeltbindinger mellem carbonatomer. 
Normalt skal der minimum være 8 konjugerede dobbeltbindinger for at der opstår farve (se side 178 B).

\paragraph{Chromofore grupper} er en anden ting der kan ændre farven på et molekyle. 
Disse er grupper med et eller to ledige elektronpar, der derfor giver anledning til en forskydning af elektronskyen. 
Når disse grupper er til stede, kan der opstå farve med færre konjugerede dobbeltbindinger (se side 180 B).
Disse er eksempelvis:
\begin{figure}[h]
    \centering
    \includegraphics[scale=0.7]{Figurer/chromoforeGrupper}
    \caption{Eksempler på chromofore grupper. Fra side 180 B}
\end{figure}


\paragraph{Auxochrome (farvemodificerende) grupper} kan ændre farven og intensiteten af farven på et organisk molekyle. 
Disse grupper er eksempelvis:
\begin{figure}[h]
    \centering
    \includegraphics[scale=.7]{Figurer/auxochromeGrupper}
    \caption{Eksempler på auxochrome grupper. Fra side 180 B.}
\end{figure}
På side 181 B ses hvordan indigo med tilførsel af 2 $-Br$ grupper bliver til purpur.


\subsection{Måling af farve}
\subsubsection{Absorbans}
Absorbans beskriver hvor meget et stof absorberer af en bestemt farve.
Vi måler arbsorbans som andelen af en intensitet der bliver absorberet af en opløsning på følgende måde:
$$A=\log(\frac{I_0}{I})$$
Her beskriver $A$ absorbansen, $I_0$ den intensitet der slipper igennem opløsningsmidlet alene og $I$ den intensitet der slipper igennem af opløsningen.
De illustreres meget godt på følgende figur:
\begin{figure}[h]
    \centering
    \includegraphics[scale=0.4]{Figurer/intensitet}
    \caption{Intensiteten $I_0$ er mindre da en del af den bliver absorberet i den farvede opløsning. Side 184 B.}
\end{figure}

\subsubsection{Lambert-Beers lov}
Beskrivelse af Lambert-Beers lov. 
$$ A= \varepsilon_\lambda \cdot [S] \cdot l  $$
$A$ er absorbans, $[S]$ er aktuel stofmængdekoncentration af stof S, $l$ er kuvettebredden og $= \varepsilon_\lambda$ er arbsorbtionskoefficenten.

\subsection{Bestemmelse af farvestofindhold i sodavand}
\subsubsection{Et enkelt farvestof}
\begin{enumerate}
    \item Bestem den maksimale absorbtionsbølgelængde (herfra kaldet $\lambda_{max}$ ved at lave et fuldspektrum.
    
    \item Bestem $\varepsilon_{\lambda_{max}}$) ved at måle sammenhørende værdier af $[S]$ og $A$ med et kendt $l$.
    
    \item Bestem $A$ ved $\lambda_{max}$ en kendt koncentration af sodavand og udnyt den kendte værdi af $\varepsilon_{\lambda_{max}}$ til at bestemme koncentrationen af favestoffet.  
\end{enumerate}

\subsection{To farvestoffer}
Gennemgås ikke med mindre at de spørger om det.
